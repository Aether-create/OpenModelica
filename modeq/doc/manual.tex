\documentclass{report}
\usepackage{verbatim}
\title{Manual - Open Source Modelica Compiler and Interactive Environment}
\author{Programming Environments Laboratory (PELAB), \\ Department of Computer and Information Science \\ Link\"{o}ping University, Sweden}
%\newcommand{\func}[1]{\item{#1}}
%\newcommand{\funcbegin}{\begin{itemize}}
%\newcommand{\funcend}{\end{itemize}}
%\newcommand{\func}[1]{\\ \hline \bf{#1} & }
%\newcommand{\funcbegin}{\begin{tabular}{|l|l} }
%\newcommand{\funcend}{\end{tabular}}
\newcommand{\func}[1]{\section{#1}}
\newcommand{\funcbegin}{}
\newcommand{\funcend}{}

\begin{document}
\maketitle
\chapter{Introduction}
This is the manual for the Open Source Modelica compiler and
interactive environment. The compiler consists of the binary file
\emph{modeq}. It can either be executed as a standard compiler if it
is given a filename as argument. But the compiler can also be started
in an interactive mode, where it listens on port no 29500 on the
localhost for a (one) incomming connection. This is how the compiler
is used togheter with the mosh (Modelica shell) binary.

For debugging purposes it is possible to start mosh and modeq in
interactive mode in two separate windows. This can be done by the
follwing commands:
\begin{verbatim}
cd modelica/mosh/src;./mosh -noserv
\end{verbatim}
and in another shell:
\begin{verbatim}
cd modelica/modeq/; ./modeq +d=interactive 
\end{verbatim}

In the following chapters we describe the builtin functions that are
available for use when running the interactive session using mosh.

\chapter{Functions for Interactive use}
This section describes a set of functions to be used in the
interactive environment, e.g.\ for loading and saving models,
translating models, and eventually simulation and plotting of results.
In the interactive environment type names and component references
(e.g.\ variables) are given as parameters to these functions \emph{as
they are}. That is, a type name is written for instance like:
\begin{verbatim}
translateModel(Modelica.Electrical.Analog.Basic.Resistor)
\end{verbatim}
and a variable is written similarly like:
\begin{verbatim}
plot({R1.i,C1.i})
\end{verbatim}
The types of these two expressions are {\tt TypeName} and {\tt
VariableName}, and special automatic quoting mechanisms are applied
when the expected type of an expression is any of these types.
For instance,
\begin{verbatim}
translateModel(A.B)
\end{verbatim}
is automatically quoted (since the expected type of the parameter is
{\tt TypeName}, and is therefore equivalent to
\begin{verbatim}
translateModel(Code(A.B))
\end{verbatim}
where {\tt Code} is the quote identifier.

Below follows the description of each predefined function in the
interactive environment.
\funcbegin
\func{translateModel(\emph{className})}
Instantiates a class and return a string
containing the flat class definition. 

The function header is defined as:
\begin{verbatim}
function translateModel
    input TypeName className;
    output String flatclass;
...
end translateModel;
\end{verbatim}

\func{list()}
Returns a string containing all class definitions.

The function header is defined as:
\begin{verbatim}
function list
    output String classes;
...
end list;
\end{verbatim}

\func{list(\emph{className})}
Returns a string containing the class definition of
the named class.

The function header is defined as:
\begin{verbatim}
function list
    input TypeName className;
    output String class;
...
end list;
\end{verbatim}

\func{typeOf(\emph{variable})}
Returns the type of the variable as a string.

The function header is defined as:
\begin{verbatim}
function typeOf
    input VariableName variable;
    output String type;
...
end typeOf;
\end{verbatim}
\func{clear()} 
Clears the complete session, as if the environment was restarted.

The function header is defined as:
\begin{verbatim}
function clear
    output Boolean res=true;
...
end clear;
\end{verbatim}
\func{clearVariables()} Clears the variable definitions that have been defined in the interactive environment.

The function header is defined as:
\begin{verbatim}
function clearVariables
    output Boolean res=true;
...
end clearVariables;
\end{verbatim}
\func{clearCache()} 
Clears the cache of flat Modelica classes.

The function header is defined as:
\begin{verbatim}
function clearCache
    output Boolean res=true;
...
end clearCache;
\end{verbatim}
\func{timing(\emph{expr})} 
Evaluates expression and returns the number of seconds
the evaluation took.

The function header is defined as:
\begin{verbatim}
function timing
    input Code expression; // Code is the supertype of
                           // any AST type, like TypeName 
                           // or VariableName.
    output Real time;
...
end timing;
\end{verbatim}
\func{cd()} 
Return the current directory as a string.

The function header is defined as:
\begin{verbatim}
function cd
    output String dir;
...
end cd;
\end{verbatim}
\func{cd(\emph{dir})} 
Change directory to the directory given as string, Returns true if successful.

The function header is defined as:
\begin{verbatim}
function cd
    input String dir;
    output Boolean res;
...
end cd;
\end{verbatim}

\func{system(\emph{str})} 
Execute \emph{str} as a system(shell) command, return integer success value.

The function header is defined as:
\begin{verbatim}
function system
    input String dir;
    output Integer res;
...
end system;
\end{verbatim}

\func{readFile(\emph{file})} 
Load \emph{file} and return a string of the
file content.

The function header is defined as:
\begin{verbatim}
function readFile
    input String file;
    output String content;
...
end readFile;
\end{verbatim}

\func{listVariables()} Return a vector of variable names that have been
defined in the interactive environment.

The function header is defined as:
\begin{verbatim}
function listVariables
    output VariableName variables[:];
...
end listVariables;
\end{verbatim}
\func{runScript(\emph{str})} Exectute script file given as string argument.

The function header is defined as:
\begin{verbatim}
function runScript
    input String scriptFile;
    output Boolean res;
...
end runScript;
\end{verbatim}

\func{loadModel(\emph{classname})}  
Load \emph{classname} from the MODELICAPATH enviroment variable,
returns true if successful.

The function header is defined as:
\begin{verbatim}
function loadModel
    input TypeName class;
    output Boolean res;
...
end loadModel;
\end{verbatim}

\func{loadFile(\emph{filename)}} Load the modelica file given as string argument.

The function header is defined as:
\begin{verbatim}
function loadFile
    input String fileName;
    output Boolean res;
...
end loadFile;
\end{verbatim}

\func{saveModel(\emph{filename},\emph{className})} 
Save the class with name \emph{className} in the file \emph{filename}.

The function header is defined as:
\begin{verbatim}
function saveFile
    input String fileName;
    input TypeName className;
    output Boolean res;
...
end saveFile;
\end{verbatim}
\func{help()} Print a short heltext of the most commonly used
commands.

The function header is defined as:
\begin{verbatim}
function help
    output String helpText;
...
end help;
\end{verbatim}
\funcend

\chapter{Functions for Modelica reflection}
This section contain functions for retrieving model information, such
as units of variables, etc.

\funcbegin

\func{getUnit(\emph{variableName, className})} Retrieve the unit of
the variable named \emph{variableName} in the class named
\emph{className}. This function will instantiate the model in order to
retrieve the unit of the given variable. It will also store the
translated (flat) class in a cache so that consecutive calls can be
made faster (e.g.\ when retrieving the units of all variables in a
class). The cache can be emptied by issuing the clearCache()
function. 

The function header is defined as:
\begin{verbatim}
function getUnit
    input VariableName variable;
    input TypeName class;
    output String unit;
...
end getUnit;
\end{verbatim}

\func{getQuantity(\emph{variableName, className})}
Retrieve the quantity of the variable named \emph{variableName} in the
class named \emph{className}.

The function header is defined as:
\begin{verbatim}
function getQuantity
    input VariableName variable;
    input TypeName class;
    output String quantity;
...
end getQuantity;
\end{verbatim}

\func{getDisplayUnit(\emph{variableName, className})}
Retrieve the displayUnit of the variable named \emph{variableName} in the
class named \emph{className}.

The function header is defined as:
\begin{verbatim}
function getDisplayUnit
    input VariableName variable;
    input TypeName class;
    output String displayUnit;
...
end getDisplayUnit;
\end{verbatim}

\func{getMin(\emph{variableName, className})}
Retrieve the min value of the variable named \emph{variableName} in the
class named \emph{className}.

The function header is defined as:
\begin{verbatim}
function getMin
    input VariableName variable;
    input TypeName class;
    output Real min;
...
end getMin;
\end{verbatim}

\func{getMax(\emph{variableName, className})}
Retrieve the max value of the variable named \emph{variableName} in the
class named \emph{className}.

The function header is defined as:
\begin{verbatim}
function getMax
    input VariableName variable;
    input TypeName class;
    output Real max;
...
end getMax;
\end{verbatim}

\func{getStart(\emph{variableName, className})}
Retrieve the start value of the variable named \emph{variableName} in the
class named \emph{className}. \emph{\bf{Note:} If the start value is given
as an initial equation, it is currently not retrievable.}

The function header is defined as:
\begin{verbatim}
function getStart
    input VariableName variable;
    input TypeName class;
    output Real start;
...
end getStart;
\end{verbatim}

\func{getFixed(\emph{variableName, className})}
Retrieve the fixed value of the variable named \emph{variableName} in the
class named \emph{className}. 

The function header is defined as:
\begin{verbatim}
function getFixed
    input VariableName variable;
    input TypeName class;
    output Boolean fixed;
...
end getFixed;
\end{verbatim}

\func{getNominal(\emph{variableName, className})}
Retrieve the nominal value of the variable named \emph{variableName} in the
class named \emph{className}. 

The function header is defined as:
\begin{verbatim}
function getNominal
    input VariableName variable;
    input TypeName class;
    output Real nominal;
...
end getNominal;
\end{verbatim}

\func{getStateSelect(\emph{variableName, className})}
Retrieve the StateSelect value of the variable named
\emph{variableName} in the class named \emph{className}. StateSelect
is defined as (According to the Modelica Language Specification)
\begin{verbatim}
type StateSelect = 
  enumeration(never, avoid, default, prefer, always);
\end{verbatim}

The function header is defined as:
\begin{verbatim}
function getStateSelect
    input VariableName variable;
    input TypeName class;
    output StateSelect stateSelect;
...
end getStateSelect;
\end{verbatim}

\funcend

\chapter{Commands for model building}
It is possible to send commands to the interactive environment for
builing and browsing models, typically used by e.g.\ a model
editor. These commands are allmost as modelica functions like the ones
above, except that they can not be used in subexpressions. They must
therefore be issued directly at the command prompt. \footnote{The reason for
this is that they are captured by the compiler before semantic checks
like type checking etc are preformed, because some of the functions
does not return true Modelica values. However, the goal is when the
Code quotation and its corresponding type system has been completely
designed to introduce these functions as Modelica functions with side
effect.} 
\footnote{These functions can also be given in a script file.}

\funcbegin

\func{createModel(\emph{modelName})}
Creates an empty model with the name \emph{modelName}.

\func{deleteModel(\emph{modelName})}
Deletes the model with the name \emph{modelName}.

\func{addComponent(\emph{compName, typeName, className, options})}
Adds a component named \emph{compName} with the type \emph{typeName}
to the class named \emph{className}. Additionally three optional named
arguments can be given:
\begin{itemize}
\item{\emph{annotate}} To give a graphical annotation to the
component. For instance
\begin{verbatim}
addComponent(R,Resistor,test, 
        annotate=Placement(transformation(x=10,y=10)));
\end{verbatim}
\item{\emph{comment}} To give a string comment to the component.
For instance
\begin{verbatim}
addComponent(C1,Capacitor,test,
        comment="The capacitor in the RC circuit");
\end{verbatim}
\item{\emph{binding}} To give the variable a binding.
For instance
\begin{verbatim}
addComponent(C1,Capacitor,test,
        binding=Code(C1=0.001));
\end{verbatim}
\end{itemize}

\func{updateComponent(\emph{compName, typeName, className, options})}
Updates the already existing component with e.g.\ a new annotation or
comment or binding.  

\func{deleteComponent(\emph{compName, className})}
Delete the component named \emph{compName} in the class named
\emph{className}.

\func{addClassAnnotation(\emph{className,options})}
Adds the named argument \emph{annotate=expression} as a class
annotation to the class named \emph{className}.

\func{getComponentCount(\emph{className})}
Returns the number of components of the class named \emph{className}.

\func{getNthComponent(\emph{className, n})}
Returns the \emph{n}:th component of the class named
\emph{className}. The return value is a \emph{vector} of the component
name, its type and the optional comment of the component.

\func{getNthComponentAnnotation(\emph{className, n})}
Returns the \emph{n}:th component annotation of the class named
\emph{className}. The return value is a \emph{vector} of the values of
the flat record for the Placement record, defined in the Modelica
Language Specification.

\func{getInheritanceCount(\emph{className})}
Returns the number of inherited classes of the class named
\emph{className}.

\func{getNthInheritedClass(\emph{className, n}}
Returns the type name of the \emph{n}:th inhertited class.

\func{getConnectionCount(\emph{className})}
Returns the number of connections in the class named \emph{className}.

\func{getNthConnection(\emph{className, n})}
Returns the names of the \emph{n}:th connection of the class named \emph{className}.

\func{getNthConnectionAnnotation(\emph{className, n})}
Returns the \emph{n}:th connection annotation of the class named
\emph{className}.

\func{addConnection(\emph{c1,c2,className,options})}
Adds the connection between connectors \emph{c1} and \emph{c2} in the
class named \emph{className}, with the optional named argument
\emph{annotate}, giving the graphical connection annotation.

\func{updateConnection(\emph{c1,c2,className,options})}
Updates an already existing connection with a new graphical
annotation.

\func{addEquation(\emph{className, expr1, expr2})}
Adds the equation \emph{expr1 = expr2} to the class named
\emph{className}. 

\func{getEquationCount(\emph{className})}
Returns the number of equations in the class named \emph{className}.

\func{getNthEquation(\emph{className, n})}
Returns the \emph{n}:th equation of the class named \emph{className}.

\func{deleteNthEquation(\emph{className, n})}
Deletes the \emph{n}:th equation of the class named \emph{className}.

\func{getConnectorCount(\emph{className})}
Returns the number of connector instances of the class named
\emph{className}. \emph{\bf{Note:} This function will perform a partial
instanitation of the class in order to find all connector instances.}

\func{getNthConnector(\emph{className, n})}
Returns the name of the \emph{n}:th connector in the class named
\emph{className}.

\func{getNthConnectorIconAnnotation(\emph{className, n})}
Returns the \emph{n}:th connector icon layer annotation.

\func{getNthConnectorDiagramAnnotation(\emph{className, n})}
Returns the \emph{n}:th connector diagram layer annotation.

\func{getIconAnnotation(\emph{className})}
Returns the Icon annotation of the class named \emph{className}.

\func{getIconAnnotation(\emph{className})}
Returns the Diagram annotation of the class named \emph{className}.

\func{getPackages(\emph{className})}
Returns the names of all packages in the class named \emph{className}.

\func{getClassNames(\emph{className})}
Returns the names of all local classes in the class named \emph{className}.

\func{getClassNames()}
Returns the names of all classes defined at the top level (global
scope).

\func{isType(\emph{className})}
Returns true if class is a type, false otherwise.

\func{isPrimitive(\emph{className})}
Returns true if class is a primitive type, false otherwise.

\func{isConnector(\emph{className})}
Returns true if class is a connector, false otherwise.

\func{isModel(\emph{className})}
Returns true if class is a model, false otherwise.

\func{isRecord(\emph{className})}
Returns true if class is a record, false otherwise.

\func{isBlock(\emph{className})}
Returns true if class is a block, false otherwise.

\func{isFunction(\emph{className})}
Returns true if class is a function, false otherwise.

\func{isPackage(\emph{className})}
Returns true if class is a package, false otherwise.

\func{existClass(\emph{className})}
Returns true if class is defined, false otherwise.

\func{existModel(\emph{className})}
Returns true if class is defined and also is a model, false otherwise.

\func{existPackage(\emph{className})}
Returns true if class is defined and also is a package, false otherwise.

\func{isParameter(\emph{componentName, className})}
Returns true if component named \emph{componentName} in the class
named \emph{className} is a parameter.

\func{isConstant(\emph{componentName, className})}
Returns true if component named \emph{componentName} in the class
named \emph{className} is a constant.

\funcend

\end{document}